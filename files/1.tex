\Section{Архитектуры компьютерных сетей}{Лекция 1}{Игорь Смирнов}

\Subsection{Эталонная модель ISO/OSI}

Компьютерные сети существуют давно (с 50-60 годов прошлого века), а потом появилось миллион протоколов и люди поняли, что невозможно связывать между собой различные системы. Тогда организация ISO создала стандарт OSI~--- эталонная модель взаимодействия открытых систем. Задачи поделены на 7 уровней. Для каждого уровня описаны функции и элементы данных, котороми компьютеры обмениваются на этом уровне.

TODO: картинка

Иногда под физическим уровнем рисуют физическую среду, а над прикладным приложение/пользователя.

Эта стандартизация позволяет сравнивать между собой разные архитектуры и понимать, что за протокол используется и какая у него функциональная нагрузка.

\begin{enumerate}

\item Физический уровень позволяет соединить между собой два устройства и передать минимальную единицу информации~--- бит. Обычно объединён с канальным уровнем и говорят про протоколы канльного-физического уровня.

\item Канальный объединяет между собой элементы компьютерной сети и позволяет двум узлам, находящимся в одной компьютерной сети обмениваться между собой потоками информации. Элемент информации~--- кадр (фрейм). Это последовательность бит, которая реализует канальный протокол. На канальном уровне могут обмениваться информацией только устройства, непосредственно соединённые канальной средой (проводной/оптической или беспроводной). То есть устройства видят друг друга без посредников. 

Примеры протоколов канального уровня: 
\begin{itemize}
    \item Ethernet (IEEE 802.3)
    \item Wi-Fi (IEEE 802.11)
\end{itemize}

\item Сетевой уровень предназначен для обмена информацией в многосегментной сети, где два конкретных узла могут быть не связаны между собой. Единица информации~--- дейтаграмма (пакет). Здесь реализуется функция аддресации. Не обеспечивает никакой надёжности.

Примеры протоколов сетевого уровня:
\begin{itemize}
    \item IPv4
    \item IPv6
    \item ICMP
\end{itemize}

\item Транспортный уровень отвечает за транспортировку информации между приложениями. Раньше говорили про связь между узлами, а сейчас о конкретных приложениях. Здесь реализуется функция аддресации приложений. Он обеспечивает определённый уровень надёжности транспортировки данных от одного приложения к другому. Элемент информации~--- пакет (иногда называют сегмент, иногда дейтаграмма).

Примеры протоколов транспортного уровня:
\begin{itemize}
    \item TCP
    \item UDP
\end{itemize}

\item Сеансовый уровень ко всемы предыдущему добавляет возможность установления и разрыва логического канала между двумя участниками соединения. Специального названия у элемента информации нет. Если как-то и называют, то просто пакетом.

\item Уровень представления предназначен для согласования формата и кодировок информации с двух сторон. Например, общие алгоритмы сжатия информации, общие алгоритмы кодирования, представление в национальных языках. Обычно протоколов уровня исполнения в последнее время не бывает, а если и бывает, то элемент информации~--- пакет.

\item Прикладной уровень~--- реализация конкретных прикладных функций. Передача файла, передача электронной почты... Элемент информации~--- либо пакет, либо сообщение.

Примеры прикладных протоколов:
\begin{itemize}
    \item HTTP
    \item FTP
    \item ...
\end{itemize}

\end{enumerate}

Мы не будем заниматься 3-7 уровнями. 1 и 2 уровни реализуются аппаратно. 3-7~--- программно.