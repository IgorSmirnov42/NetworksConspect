\Section{Конфигурирование в компьютерных сетях}{Лекция 7}{Игорь Смирнов}

Есть два подхода к управлению сетевыми параметрами: ручная настройка и автоматизированная настройка. 

Когда мы что-то настраиваем автоматизированно, мы хотим скопом настроить несколько параметров (как минимум, IP-адрес, как максимум, ещё несколько десятков параметров).

Существует три стандартных способа автоматизированной настройки:
\begin{itemize}
    \item Пртокол RARP
    \item Протокол BOOTP
    \item Протокол DHCP
\end{itemize}

\Subsection{Протоколы RARP и BOOTP}

RARP~--- совсем раритетный протокол.

Клиент без адреса обращался к серверу,а в ответ получал IP-адрес по MAC-адресу из таблицы RARP-сервера.

Использовался только для получения IP-адреса. Сейчас не используется.

BOOTP~--- долгое время был стандартом для динамического конфигурирования.

Тоже клиент-серверный подход. Есть BOOTP-сервер, который умеет по запросам раздавать необходимые параметры:
\begin{itemize}
    \item IP-адрес
    \item Маску сети
    \item Маршрутизатор <<по умолчанию>>
\end{itemize}

BOOTP позволяет иметь цепочку ретрансляторов. То есть можно иметь один сервер на несколько сегментов сети, а запросы между сегментами пересылать через ретранслятор.

Работал так же, как и RARP: администратор руками настраивал таблицу соответствия MAC-адресов и параметров узла.

BOOTP сейчас считается устаревшим.

\Subsection{Протокол DHCP}

Это чуть ли не единственный протокол, разработанный компанией Microsoft.

Цель~--- расширить количество параметров из BOOTP и расширить количество способов, как динамически управлять данными, которые запрашивают адреса.

BOOTP нас, помимо параметров, не устраивает те